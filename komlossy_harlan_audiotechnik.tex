
\documentclass{acm_proc_article-sp}
\usepackage[utf8]{inputenc}
\usepackage{ngerman}

\begin{document}

\title{Feststellung und Bewertung des Melodie- und Rhythmusgedächtnisses}
\subtitle{Lehrgebiet Audiotechnik SS2015 31.07.2015}
%\subtitle{Lehrverantwortliche: Dr. rer. nat. Dieter Kemter, PD Dr.-Ing. habil. Günther Schatter }

\numberofauthors{2} 
\author{
% 1st. author
\alignauthor
Kristof Komlossy\\
       \affaddr{Bauhaus-Universität Weimar}\\
       \affaddr{Fakultät Medien}\\
       \email{kristof.komlossy@uni-weimar.de}
% 2nd. author
\alignauthor
Jakob Harlan\\
       \affaddr{Bauhaus-Universität Weimar}\\
       \affaddr{Fakultät Medien}\\
       \email{jakob.harlan@uni-weimar.de}
} %author

\maketitle

\begin{abstract}

\end{abstract}

\keywords{Rhytmusgedächnis, Meldodiegedächnis, Max/MSP} % NOT required for Proceedings

\section{Einleitung}

\section{Verwante Arbeiten}

\section{Konzept}
\section{Implementierung}
\section{Studie}
\subsection{Design}
\subsection{Durchführung}
\subsection{Auswertung}
\subsection{Diskussion}
\section{Fazit}
\end{document}
