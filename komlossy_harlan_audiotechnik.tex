\documentclass{acm_proc_article-sp}
\usepackage[utf8]{inputenc}
\usepackage[german]{babel}
\usepackage[backend=bibtex,urldate=long]{biblatex}
\usepackage{color}
\addbibresource{lit.bib}
\begin{document}

\title{Feststellung und Bewertung des Melodie- und Rhythmusgedächtnisses}
\subtitle{Lehrgebiet Audiotechnik SS2015 31.07.2015}
%\subtitle{Lehrverantwortliche: Dr. rer. nat. Dieter Kemter, PD Dr.-Ing. habil. Günther Schatter }

\numberofauthors{2} 
\author{
% 1st. author
\alignauthor
Kristof Komlossy\\
       \affaddr{Bauhaus-Universität Weimar}\\
       \affaddr{Fakultät Medien}\\
       \email{kristof.komlossy@uni-weimar.de}
% 2nd. author
\alignauthor
Jakob Harlan\\
       \affaddr{Bauhaus-Universität Weimar}\\
       \affaddr{Fakultät Medien}\\
       \email{jakob.harlan@uni-weimar.de}
} %author

\maketitle

\begin{abstract}
In dieser Arbeit werden Fragestellungen zu musikalischen Fähigkeiten beantwortet. Dazu haben wir Programme zum Testen des Melodiegedächtnisses und des Rhythmusgedächtnisses entwickelt und in einer Nutzerstudie eingesetzt. Die Programme und die Ergebnisse der Nutzerstudie werden wir in der vorliegenden Arbeit vorstellen und auswerten.
\end{abstract}

\keywords{Rhytmusgedächnis, Melodiegedächnis, Max/MSP} % NOT required for Proceedings

\section{Einleitung}
Wir haben uns mit den Fragestellungen \textcolor{red}{vielzurauswah} beschäftigt. 
Zu Beginn (2) werden wir die bereits existierenden und für uns relevanten musikalischen Tests vorstellen.\\
Auf diese bereits vorhandenen Untersuchungen haben wir unsere Vorüberlegungen aufgebaut (3) und haben je ein Programm zum Testen von zwei ähnlichen musikalischen Fähigkeiten entwickelt. Zum Einen haben wir ein Programm zum Testen des Melodiegedächtnisses entwickelt. Dieses Programm spielt dem Nutzer eine Melodie vor. Anschließend muss der Nutzer die Melodie aus dem Gedächtnis rekapitulieren können und dabei die relativen Tonhöhen bestimmen können.
Die zweite Fähigkeit die wir getestet haben ist das Rhythmusgedächtnis. Ähnlich wie bei dem Melodiegedächtnis spielt das Programm dem Nutzer einen Rhythmus vor, welchen der Nutzer dann ebenfalls aus dem Gedächtnis wiedergeben muss. Da die Rhythmen aus drei verschiedenen Tonlängen zusammengesetzt sind muss der Nutzer angeben welche Länge jeder einzelne Ton hatte. Auf die nähere Implementierung dieser Programme werden wir in Kapitel 4 eingehen.\\
Um unsere zu Beginn genannten Fragestellungen zu beantworten haben wir mit unseren beiden Programmen eine Nutzerstudie durchgeführt. Die Beschreibung der Studie und die Auswertung der Ergebnisse werden wir in Kapitel 5 vornehmen.
\section{Verwandte Arbeiten}
Im Jahr 1919 hat der amerikanische Psychologe Carl Emil Seashore eine Reihe von Tests zur Ermittlung musikalischer Begabung entwickelt\cite{gordon:2000}. Seashore hat für diese Arbeit fast 20 Jahre das Thema der Charakterisierung und Beschreibung musikalischer Begabung untersucht. Das Ergebnis waren erstmalig standardisierte Tests zur Ermittlung musikalischer Fähigkeiten. Dabei fokussierte sich Seashore auf das Gedächtnis für Ton- und Rhythmusfolgen und die Unterscheidungsfähigkeit folgender Eigenschaften: Tonhöhen, Lautstärken, Rhythmen, Tonlängen, und Klangfarben.\\
Aufgrund der Standardisierung der Seashore-Tests und der grundlegenden Forschung vor Ihrer Entwicklung wurden die Tests häufig von anderen musikalischen Tests als Grundlage genutzt. 
\section{Konzept}

\section{Implementierung}

\section{Studie}
In diesem Kapitel wird die Konzeption, Durchführung, qualitative und statistische Auswertung der von uns durchgeführten Studie vorgestellt.  \textcolor{red}{BEOBACHTUNGEN} Zuletzt werden die Vorgehensweisen und Ergebnisse diskutiert.

\subsection{Design und Fragestellungen}
In unserer Studie sollte von möglichst vielen Testpersonen sowohl das Melodie- als auch das Rhythmusgedächtnis getestet werden. Dafür haben wir die entwickelten Programme verwendet.\\
Außerdem sollte die Studie Auskunft darüber geben welche Faktoren zu einer guten Leistung in den getesteten Bereichen führt. Hierfür erhoben wir folgende Informationen über die Teilnehmer:\\
\begin{itemize} 
\item Alter
\item Geschlecht
\item Musikalische Vorkenntnisse
\item Empfundene Schwierigkeit bei beiden Tests
\item Abspielanzahl der einzelnen Samples
\end{itemize}
Die zentrale Hypothese die es zu validieren galt war dass musikalische Vorkenntnisse hilfreich bei der Bewältigung der Aufgaben sind. \\
Auch ob Alter oder Geschlecht einen messbaren Unterschied machen interessierte uns. Ebenfalls haben wir untersucht ob die empfundene Schwierigkeit mit dem tatsächlichen Ergebnis der Kandidaten korreliert. 
Außerdem kann versucht werden Unterschiede in der Erfolgsrate zwischen Rhythmus- und Melodiegedächnis zu finden. 
% War der 2te Test erfolgreicher?
\subsection{Durchführung}

\subsubsection{Ablauf}
Um die nötigen Informationen zu erheben musste vor dem eigentlichen Test ein Fragebogen beantwortet werden, welcher Alter, Geschlecht und musikalische Vorkenntnisse abfragt. Die musikalischen Vorkenntnisse haben die Kandidaten auf einer Skala von 1-10 selber bewertet. Allerdings haben hier die begleitenden Studienautoren beratend eingegriffen um Vergleichbarkeit zu gewährleisten.\\
Darauf folgend wurde der Testperson die Programmoberfläche und der Ablauf der Tests erklärt.\\ 
Die Kandidaten mussten jeweils eine Testreihe zum Rhythmusgedächtnis und eine zum Melodiegedächnis absolvieren. Die Reihenfolge hat wurde alternierend gewählt um Lerneffekte auszuschließen. \\
%Test reihe selber hier?
Nach den Tests mussten die Kandidaten noch einen weiteren kurzen Fragebogen ausfüllen in dem aufgenommen wurde wie schwer sie die einzelnen Tests gefunden haben. Außerdem stand hier ein Textfeld für freie Kommentare zur Verfügung. 
\subsubsection{Teilnehmer}
Die Studie wurde mit 23 Teilnehmern durchgeführt.\\
Die meisten wurden aus unserem studentischen Umfeld akquiriert, was die unrepräsentative Altersverteilung, mit einer deutlichen Häufung im Bereich 20-25 Jahre, erklärt. Die davon abweichenden Kandidaten sind Familienmitglieder und andere Bekannte.
\begin{itemize} 
\item \textcolor{red}{BILD DER ALTERSVERTEILUNG}
\end{itemize}
Ungefähr die Hälfte der Teilnehmer stammt aus dem Bereich der Informatik. Die anderen verteilen sich auf weitere Studienfächer, Schüler und Berufstätige.\\
Es ist uns gelungen ungefähr gleich viele männliche wie weibliche Studienteilnehmer zu testen. 
\begin{itemize} 
\item \textcolor{red}{BILD DER GESCHLECHTERVERTEILUNG}
\end{itemize}
\subsection{Auswertung}
\subsubsection{Qualitative}
\subsubsection{Statistische}
Die aufgestellte Hypothese das musikalische Vorkenntnisse das Melodie- und Rhythmusgedächnis stärken wird von den Ergebnissen unserer Studie bestätigt.
\begin{itemize} 
\item \textcolor{red}{SCATTERPLOT Vorkenntnisse -> Erfolgsrate Rhtym und Melodie}
\end{itemize}
Der visuelle Eindruck einer Korrelation wird vom linearen Korrelationseffizienz nach Pearsons \textcolor{red}{(Quelle)} \textcolor{blue}{(0.52 für Melodie, 0.71 für Rhythmus)} unterstützt. Und auch der Rangkorrelationskoeffizient nach Spearman \textcolor{red}{(Quelle)} \textcolor{blue}{(0.54 für Melodie, 0.84 für Rhythmus)} widerspricht diesem nicht.
%http://www.socscistatistics.com/tests/spearman/Default2.aspx
\subsection{Beobachtungen}
Beim Begleiten der Testpersonen durch die Studie konnten wir noch Interessantes abseits der erhobenen Daten beobachten, worauf hier eingegangen werden soll.\\
Wir haben zwischen den Kandidaten unterschiedliche Vorgehensweisen beim Merken der Samples erkannt. Insbesondere die Frequenz des Abspielens variierte stark. Es gab Kandidaten die haben sich ein Stück nur zweimal angehört. Zwischen den \textcolor{green}{Abspielungen} haben sie versucht sich an das Gehörte zu erinnern und sich selber mental vorzuspielen. Beim zweiten mal haben sie ihre Vorstellung und die Wirklichkeit verglichen. Dies hat ihnen gereicht um sich sicher zu fühlen das Sample korrekt wiedergeben zu können.\\
Andere dagegen haben die Beispiele sobald sie endeten wieder abgespielt und keine Pausen entstehen lassen. Dieser Ansatz hat Abspielhäufigkeiten von \textcolor{blue}{10} Wiederholungen und mehr verursacht.\\ 
Was zu diesen unterschiedlichen Ansätzen zum Merken führt und ob diese verschieden gute Ergebnisse erreichen kann mit unserer Studie nicht beantwortet werden.
\subsection{Diskussion}
%TODO
%Transfer Gemerktes -> Notation gemessen, nicht das gedächnis selber\\
%Zuviele unregelmäßig ändernde Parameter: Alter, Studiengang, ..\\
%Vergleichbarkeit Rhythmus <-> Melodie\\
%Auswahl der Samples -> nicht random \\


\section{Fazit}
\printbibliography
\end{document}
