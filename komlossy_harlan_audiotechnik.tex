
\documentclass{acm_proc_article-sp}
\usepackage[utf8]{inputenc}
\usepackage{ngerman}

\begin{document}

\title{Feststellung und Bewertung des Melodie- und Rhythmusgedächtnisses}
\subtitle{Lehrgebiet Audiotechnik SS2015 31.07.2015}
%\subtitle{Lehrverantwortliche: Dr. rer. nat. Dieter Kemter, PD Dr.-Ing. habil. Günther Schatter }

\numberofauthors{2} 
\author{
% 1st. author
\alignauthor
Kristof Komlossy\\
       \affaddr{Bauhaus-Universität Weimar}\\
       \affaddr{Fakultät Medien}\\
       \email{kristof.komlossy@uni-weimar.de}
% 2nd. author
\alignauthor
Jakob Harlan\\
       \affaddr{Bauhaus-Universität Weimar}\\
       \affaddr{Fakultät Medien}\\
       \email{jakob.harlan@uni-weimar.de}
} %author

\maketitle

\begin{abstract}

\end{abstract}

\keywords{Rhytmusgedächnis, Meldodiegedächnis, Max/MSP} % NOT required for Proceedings

\section{Einleitung}

\section{Verwante Arbeiten}

\section{Konzept}
\section{Implementierung}
\section{Studie}
In diesem Kapitel wird die Konzeption, Durchführung, qualitative und statistische Auswertung der durchgeführten Studie vorgestellt. Zuletzt werden die Vorgehensweisen und Ergebnisse diskutiert.

\subsection{Design und Fragestellungen}
In unserer Studie sollte von möglichst vielen Testpersonen sowohl das Melodie- als auch das Rhythmusgedächtnis getestet werden. Dafür wurden die entwickelten Programme verwendet.\\
Außerdem sollte die Studie Auskunft darüber geben welche Faktoren zu einer guten Leistung in den getesteten Bereichen führt. Hierfür wurden einige Informationen über die Teilnehmer erhoben.\\
\begin{itemize} 
\item Alter
\item Geschlecht
\item Musikalische Vorkenntnisse
\item Empfundene Schwierigkeit bei beiden Tests
\item Abspielanzahl der einzelnen Samples
\end{itemize}
Die zentrale Hypothese die es zu validieren galt war dass musikalische Vorkenntnisse hilfreich bei der Bewältigung der Aufgaben sind. \\
Aber auch ob Alter oder Geschlecht einen messbaren Unterschied machen ist von Interesse. Ebenfalls untersucht wurde ob die empfundene Schwierigkeit mit dem tatsächlichen Ergebnis der Kandidaten korreliert. 
%TODO  Melodie gut auch Rhythmus gut ?

\subsection{Durchführung}

\subsubsection{Ablauf}
Um die nötigen Informationen zu erheben musste vor dem eigentlichen Test ein Fragebogen beantwortet werden, welcher Alter, Geschlecht und musikalische Vorkenntnisse abfragt. Die musikalischen Vorkenntnisse wurden auf einer Skala von 1-10 von den Kandidaten selber bewertet. Allerdings haben hier die begleitenden Studienautoren beratend eingegriffen um Vergleichbarkeit zu gewährleisten.\\
Darauf folgend wurde der Testperson die Programmoberfläche und der Ablauf der Tests erklärt. \\
\subsubsection{Teilnehmer}
Die Studie wurde mit 23 Teilnehmern durchgeführt. 

\subsection{Auswertung}


\subsection{Diskussion}
%TODO
%Transfer Gemerktes -> Notation gemessen, nicht das gedächnis selber\\
%Zuviele unregelmäßig ändernde Parameter: Alter, Studiengang, ..\\
%Vergleichbarkeit Rhythmus <-> Melodie\\
%Auswahl der Samples -> nicht random \\


\section{Fazit}
\end{document}
